\documentclass[12pt]{article}

\usepackage{geometry}
\usepackage[utf8]{inputenc}
\usepackage[polish]{babel}
\usepackage{polski}
\usepackage{hyperref}
\usepackage{graphicx}
\usepackage{verbatim}
\usepackage{acronym}
\usepackage{fancyhdr}
\usepackage[usenames]{color}

\hypersetup{
  linkbordercolor={1 1 1},
  urlbordercolor={1 1 1},
  colorlinks=true
}

\pagestyle{fancy}
\cfoot{}
\rfoot{\thepage}

\author{\textbf{Antoni Piechnik} \\ \\ prowadzący: \textbf{dr inż. Marek Gajęcki}}
\title{
Strukturyzacja słownika języka polskiego do pseudo-XML
\begin{figure*}[h]
  \centerline{
    \includegraphics[scale=1.0]{images/logo_agh.pdf}
  }
\end{figure*}
}

\begin{document}

\maketitle
\newpage
\tableofcontents
\newpage

\section{Wstęp}
\subsection{Wizja}
Głównym zadaniem projektu jest poznanie struktury
danych typu GIS (geographical information system), jak również analiza oraz
wykorzystanie tego typu danych w wizualizacji danych meteorologicznych. System
docelowo ma za zadanie przedstawienie sytuacji meteorologicznej na podstawie
danych zbieranych na bieżąco jak również danych historycznych zgromadzonych
poprzednio. System ma również mieć możliwość udostępniania danych/wizualizacji
historycznych na życzenie użytkownika. Do celów badania wydajności systemu
wykorzystywane będą dane z przynajmniej dwóch źródeł informacji
meteorologicznej, podczas gdy system ma domyślnie obsługiwać 4-5 stacji
narciarskich (po kilka punktów na każdą stację).

\subsection{Ocena ryzyka}
Technologia Django (w konsekwencji również Python) daje dobre perspektywy rozwoju:
Python jest językiem bogatym w biblioteki (m.in. do Oracle) i jako język dynamiczny
daje możliwość łatwego rozszerzania aplikacji. Dobrze rokuje także projekt GeoDjango
(\url{http://geodjango.org/}) w związku z czym można sądzić, że nie napotkamy na
większe problemy implementacyjne (związane z technologią).

Dane zapisywane są przez
\begin{verbatim}
file = File.open(filename, "w")
file.write(Marshal.dump(data))
\end{verbatim}
Aby je odczytać wystarczy
\begin{verbatim}
data = Marshal.load(File.read(filename))
\end{verbatim}

\section{Linki}
\begin{description}
  \item[Źródła (repozytorium Git)] \url{http://github.com/michalbugno/projekt-oszbd/}
  \item[Projekt Django] \url{http://www.djangoproject.com/}
  \item[Projekt GeoDjango] \url{http://geodjango.org/}
  \item[API GoogleMaps] \url{http://code.google.com/apis/maps/documentation/}
  \item[System kontroli wersji Git] \url{http://git-scm.com/}
  \item[Python] \url{http://www.python.org/}
  \item[Ruby] \url{http://www.ruby-lang.org/}
  \item[Oracle Spatial] \url{http://download.oracle.com/docs/cd/B10501\_01/appdev.920/a96630/toc.htm}
\end{description}

\end{document}

